%%%%%%%%%%%%%%%%%%%%%%%%%%%%%%%%%%%%%%%%%
% "ModernCV" CV and Cover Letter
% LaTeX Template
% Version 1.11 (19/6/14)
%
% This template has been downloaded from:
% http://www.LaTeXTemplates.com
%
% Original author:
% Xavier Danaux (xdanaux@gmail.com)
%
% License:
% CC BY-NC-SA 3.0 (http://creativecommons.org/licenses/by-nc-sa/3.0/)
%
% Important note:
% This template requires the moderncv.cls and .sty files to be in the same 
% directory as this .tex file. These files provide the resume style and themes 
% used for structuring the document.
%
%%%%%%%%%%%%%%%%%%%%%%%%%%%%%%%%%%%%%%%%%

%----------------------------------------------------------------------------------------
%	PACKAGES AND OTHER DOCUMENT CONFIGURATIONS
%----------------------------------------------------------------------------------------

\documentclass[11pt,a4paper,sans]{moderncv} % Font sizes: 10, 11, or 12; paper sizes: a4paper, letterpaper, a5paper, legalpaper, executivepaper or landscape; font families: sans or roman

\moderncvstyle{casual} % CV theme - options include: 'casual' (default), 'classic', 'oldstyle' and 'banking'
\moderncvcolor{blue} % CV color - options include: 'blue' (default), 'orange', 'green', 'red', 'purple', 'grey' and 'black'

\usepackage{lipsum} % Used for inserting dummy 'Lorem ipsum' text into the template

\usepackage[scale=0.75]{geometry} % Reduce document margins
%\setlength{\hintscolumnwidth}{3cm} % Uncomment to change the width of the dates column
%\setlength{\makecvtitlenamewidth}{10cm} % For the 'classic' style, uncomment to adjust the width of the space allocated to your name

%----------------------------------------------------------------------------------------
%	NAME AND CONTACT INFORMATION SECTION
%----------------------------------------------------------------------------------------

\firstname{Jerin} % Your first name
\familyname{Roberts} % Your last name

% All information in this block is optional, comment out any lines you don't need
\title{Curriculum Vitae}
\address{2165 westsyde Rd.}{Kamloops, BC v2b7c3}
\mobile{(250) 682 6536}
\phone{(250) 318 0687}
\email{robertsj@snolab.ca}
 % The first argument is the url for the clickable link, the second argument is the url displayed in the template - this allows special characters to be displayed such as the tilde in this example

%\photo[70pt][0.4pt]{pictures/picture} % The first bracket is the picture height, the second is the thickness of the frame around the picture (0pt for no frame)
%\quote{"A witty and playful quotation" - John Smith}

%----------------------------------------------------------------------------------------

\begin{document}








%----------------------------------------------------------------------------------------
%	COVER LETTER
%----------------------------------------------------------------------------------------

% To remove the cover letter, comment out this entire block

\clearpage

\recipient{Statement of Purpose: Jerin Roberts}{Office of Admissions\\Simon Fraser University\\8888 University Drive\\V5A 1S6 Burnaby, B.C.} % Letter recipient
\date{\today} % Letter date
\opening{} % Opening greeting  
\closing{} % Closing phrase
%\enclosure[Attached]{curriculum vit\ae{}} % List of enclosed documents

\makelettertitle % Print letter title

"Don't fall" my supervisor warned me, "Falling will most certainly halt the progress of this project." I began to sweat, my heart throbbing in my chest. I began to think back to the interview, how I could never imagine to have such an opportunity. "Alright lower him in." The winch began to whine as it came to life and I began to descend into the darkness.

 As my eyes adjusted to the low light, I could make out the iconic spherical structure I had remembered reading about. The metal frame felt cold through my clean suit as I docked with the structure. Once detached from my harness and safely perched my fear began to fade, replaced with curiosity. The cavity was massive, which sported the most delicate of detectors; the original SNO neutrino detector, a magnificent display of ten thousand photo-multiplyier tubes all straining to catch a glimpse of the most elusive particles. It was while gawking atop this massive geodesic sphere below 2km of rock and ore, that I knew this was the work of many great scientists and engineers. I became fascinated with complexity of the structure, imagining all the possible problems they had to overcome for this to exist. This brought such excitement and wonder, I knew I must pursue a career in both science and engineering. 

 Since high school I have always had a natural interest physics and math, as they always felt intuitive. It wasn't until beginning post secondary that I realized I wanted to pursue a related career. My early university days where not at all smooth. In addition to not being fully committed, I was also required to support my education. Without financial aid and living on my own I struggled through my first and second years of school. However Determined not to be brought down I enrolled at Thompson Rivers University (TRU) hoping to have an opportunity to give it my all.

  During my studies at TRU I acquired a great working knowledge of applied physics and a large amount of engineering related experience. For a directed studies project I designed and professionally constructed operational sensors for a high altitude balloon experiment using a circuit milling machine and CAD software. The balloon was launched and retrieved successfully attaining a maximum altitude of 123000 feet. Never in my life have I had a greater satisfaction than seeing my work briefly touch the edge of space. 

 Following my directed studies experience, I received an undergraduate research award in which I used to modify and fabricated RFID sensing equipment for animal detection. The project involved managing a budget and working with researchers to construct 25 units using surface mounted technology. The project required a great deal of patience and attention to detail in order to assure all units were precisely milled, assembled and tested to exceed the requirements. I often found it interesting to see how my work could be applied to vastly different fields. TRU has given me hands-on experience which coupled with my analytical and strategic nature will be invaluable towards my professional career.

  In the Summer of 2014 I worked at SNOLAB under the supervision of Dr. Christine Kraus on the SNO+ experiment. During my term I spent a great deal of time engineering mechanical systems for the project. Specifically addressing the slippage issues plaguing the umbilical retrieval mechanisms (URM's), a device responsible for lowering radioactive sources into the multi-million dollar detector. In addition to redesigning high-traction pulleys, I also designed and fabricated a chain-drive system which successfully met budget, space and radioactivity requirements. The results of my designs reduced slippage down to a reliable 5 percent far surpassing the original 60-70 percent and will be implemented into the final URM design. In addition to giving me a glimpse of the seemingly endless possibilities, my time with SNOLAB has also given me a great technical experience that I believe combined with my mechanical intuitiveness and knowledge of physics will aid me during my professional development and advancement within engineering.

 Although I enjoyed physics, my research experiences have really confirmed my interests in engineering. Since Fall 2014 I have been involved in the development of the M9 prototype muon spectrometer currently being designed and fabricated at TRIUMF. Under the supervision of Dr. Syd Kreitzman, I designed universal equipment used for testing specific experimental components, which involved generating professional drawings for the on-site machine shop. I also designed from scratch a sophisticated simulation used to determine the possible resolution of the new spectrometer given the geometrical configurations for scintillation pieces. I have been re-hired to continue the project during the summer of 2015. My current work at TRIUMF has enabled me to refine my engineering and programming abilities which will be an asset for my future career goals.  
 
 At Simon Fraser University, I plan to enroll in the engineering science program. In this program I hope to draw on my educational and research experience as a foundation for studying more advanced concepts. I am particularly interested in researching the ties between engineering, physics, and applied mechanics. I would regard my admission to Simon Fraser University not only as a great honor and success but also as an obligation for hard work. I look forward to starting my engineering career, as well as pursuing opportunities for professional development and advancement within Simon Fraser University.

%\makeletterclosing % Print letter signature

%----------------------------------------------------------------------------------------

\end{document}