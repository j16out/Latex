%%%%%%%%%%%%%%%%%%%%%%%%%%%%%%%%%%%%%%%%%
% "ModernCV" CV and Cover Letter
% LaTeX Template
% Version 1.11 (19/6/14)
%
% This template has been downloaded from:
% http://www.LaTeXTemplates.com
%
% Original author:
% Xavier Danaux (xdanaux@gmail.com)
%
% License:
% CC BY-NC-SA 3.0 (http://creativecommons.org/licenses/by-nc-sa/3.0/)
%
% Important note:
% This template requires the moderncv.cls and .sty files to be in the same 
% directory as this .tex file. These files provide the resume style and themes 
% used for structuring the document.
%
%%%%%%%%%%%%%%%%%%%%%%%%%%%%%%%%%%%%%%%%%

%----------------------------------------------------------------------------------------
%	PACKAGES AND OTHER DOCUMENT CONFIGURATIONS
%----------------------------------------------------------------------------------------

\documentclass[11pt,a4paper,sans]{moderncv} % Font sizes: 10, 11, or 12; paper sizes: a4paper, letterpaper, a5paper, legalpaper, executivepaper or landscape; font families: sans or roman

\moderncvstyle{casual} % CV theme - options include: 'casual' (default), 'classic', 'oldstyle' and 'banking'
\moderncvcolor{blue} % CV color - options include: 'blue' (default), 'orange', 'green', 'red', 'purple', 'grey' and 'black'

\usepackage{lipsum} % Used for inserting dummy 'Lorem ipsum' text into the template

\usepackage[scale=0.75]{geometry} % Reduce document margins
%\setlength{\hintscolumnwidth}{3cm} % Uncomment to change the width of the dates column
%\setlength{\makecvtitlenamewidth}{10cm} % For the 'classic' style, uncomment to adjust the width of the space allocated to your name

%----------------------------------------------------------------------------------------
%	NAME AND CONTACT INFORMATION SECTION
%----------------------------------------------------------------------------------------

\firstname{Jerin} % Your first name
\familyname{Roberts} % Your last name

% All information in this block is optional, comment out any lines you don't need
\title{Curriculum Vitae}
\address{2165 westsyde Rd.}{Kamloops, BC v2v7c3}
\mobile{(250) 682 6536}
\phone{(250) 318 0687}
\email{robertsj@snolab.ca}
 % The first argument is the url for the clickable link, the second argument is the url displayed in the template - this allows special characters to be displayed such as the tilde in this example

%\photo[70pt][0.4pt]{pictures/picture} % The first bracket is the picture height, the second is the thickness of the frame around the picture (0pt for no frame)
%\quote{"A witty and playful quotation" - John Smith}

%----------------------------------------------------------------------------------------

\begin{document}

\makecvtitle % Print the CV title

%----------------------------------------------------------------------------------------
%	EDUCATION SECTION
%----------------------------------------------------------------------------------------

\section{Education}

\cventry{Expected- April,2015}{Bachelors of Science}{Thompson Rivers University}{Kamloops}{\textit{GPA -- 3.6}}{Major Physics}  % Arguments not required can be left empty

%----------------------------------------------------------------------------------------
%	WORK EXPERIENCE SECTION
%----------------------------------------------------------------------------------------

\section{Experience}


\cventry{Fall 2014  --Present}{Undergraduate Research Assistant}{\textsc{TRIUMF}}{Vancouver, BC}{}{Aiding in the development of the M9 Prototype Muon Spectrometer
\newline{}
Detailed achievements:
\begin{itemize}
\item Designing Experimental Apparatus Equipment using Solidworks for SiPM tests
\item Characterizing and Analyzing SiPMs for final selection
\item Designed Positron Timing Simulation (C++/ROOT)
\item Designed scripts for Pulse Detection and Characterization from SiPM's (C++/ROOT) 
\newline{}
\end{itemize}}


\cventry{Summer 2014}{Undergraduate Research Assistant}{\textsc{SNOLAB}}{Sudbury, Ontario}{}{Fixed the umbilical retrieval mechanism (URM) for SNO+ experiment key for lowering sources into the detector. 
\newline{}
Detailed achievements:
\begin{itemize}
\item Assisted PMT assembly in class 2000 and 5000 underground clean labs
\item Designed and implemented chain drive system vastly improving performance
\item Re-Designed/fabricated Pulley Wheels using FreeCAD/3D printer
\item FEM analysis using Z88 Aurora to verify pulley integrity 
\newline{}
\end{itemize}}




\cventry{2012--Present}{Laboratory Instructor}{\textsc{Thompson Rivers University}}{Kamloops, BC}{}{Lead and taught 1st year Physics Labs
\newline{}
Detailed achievements:
\begin{itemize}
\item Preparing and Presenting Lectures for Laboratory Course
\item Instructing students on how to perform experiments
\item Working quickly to provide valuable solutions for student questions
\item Marking and grading assignments and exams 
\end{itemize}}

%------------------------------------------------


%----------------------------------------------------------------------------------------
%	AWARDS SECTION
%----------------------------------------------------------------------------------------

\section{Awards}

\cvitem{2013}{Undergraduate Research Experience Award Program (UREAP)}


%----------------------------------------------------------------------------------------
%	COMPUTER SKILLS SECTION
%----------------------------------------------------------------------------------------

\section{Computer skills}

\cvitem{Languages}{\textsc{C/C++}, \LaTeX , Bash (Linux), Assembly, \textsc R}
\cvitem{OS/Programs}{Ubuntu 14.04, Scientific Linux, Microsoft Windows, FreeCAD, Solidworks, Eagle CAD, Texmaker, Labview, Z88Aurora}
\cvitem{Devices}{Arduino, Altera FPGA, Raspberry pi, 
Newport Stepper Motors/Drivers, Tektronix (tekVISA), PICkit 2}

%----------------------------------------------------------------------------------------
%	COMMUNICATION SKILLS SECTION
%----------------------------------------------------------------------------------------
%\section{Mechanical Skills}
%Small Engine Repair
%\begin{itemize}
%\item valve adjustment (shimming + lapping)
%\item Cylinder honing + ring/piston replacement
%\item Bearing replacement (Bottom End + Transmission)
%\item Carburetor tuning/jetting
%\end{itemize}
  
%Suspension Service
%\begin{itemize}
%\item All Fox FIT cartridges (service/rebuild)
%\item Fox DHX rear shocks
%\item All Fox, RockShox, and Marzocchi Forks (service seals and internals)
%\end{itemize}
%Miscellaneous
%\begin{itemize}
%\item Surface Mount Soldering (circuit boards)
%\item Welding (wire feed and stick) and Cutting (oxy-acetylene)
%\item Drill press, Milling, and grinding buffing
%\item Tapping + Helicoiling damaged threads
%\end{itemize}

\section{Volunteer Work}

\cvitem{2010-Present}{Physics Help Center hosted by Phi-6 Club (current member)}
\cvitem{2012-2014}{Physics Magic Show Presentation}
\cvitem{2012-2013}{Open House Science Night Presenter}



\section{Hobbies and Interests}

\renewcommand{\listitemsymbol}{-~} % Changes the symbol used for lists

\cvlistdoubleitem{RC Helicopters/Aircraft}{Model Rocketry}
\cvlistdoubleitem{DH Biking}{Flight Simulation}

\section{References}

\cventry{}{Dr. Syd Kreitzman}{\textsc{TRIUMF}}{Research Scientist MuSR}{}{
syd@triumf.ca
\newline{}
604-222-7303 
}

\cventry{}{Dr. Christine Kraus}{\textsc{SNOLAB}}{Canada Research Chair in Particle Astrophysics}{}{
tine@snolab.ca
\newline{}
705-561-8413   
}

\cventry{}{Dr. Mark Paetkau}{\textsc{TRU}}{Professor Physical Sciences}{}{
mpaetkau@tru.ca
\newline{}
250-828-5453
}






%----------------------------------------------------------------------------------------
%	COVER LETTER
%----------------------------------------------------------------------------------------

% To remove the cover letter, comment out this entire block

\clearpage

\recipient{Office of Admissions}{Simon Fraser University\\8888 University Drive\\V5A 1S6 Burnaby, B.C.} % Letter recipient
\date{\today} % Letter date
\opening{Dear Sir or Madam,} % Opening greeting  
\closing{Sincerely yours,} % Closing phrase
\enclosure[Attached]{curriculum vit\ae{}} % List of enclosed documents

\makelettertitle % Print letter title
My Goal is to become a student within the prestigious institution of Simon Fraser University, while learning and gaining experience for my future goal of pursuing a career in Engineering. Given my educational background and experience will allow me to be an asset to your organization.

Since Fall 2014 I have been involved in the development of the M9 prototype beamline spectrometer currently being designed and built at TRIUMF. Under the supervision of Dr. Syd Kreitzman, I designed multi-applicable equipment used for testing specific experimental components, which involved generating professional drawings for the on-site machine shop using Solidworks. I also designed from scratch a sophisticated simulation in C++/ROOT used to help determine the possible resolution of the new spectrometer given different geometrical configurations for scintillation pieces. In Addition I wrote scripts to detect and characterize pulses during SiPM characterization phases in C++/ROOT. The project has enabled me to refine my engineering and programming abilities which will be an asset to the engineering science program. 

Summer 2014 I worked for SNOLAB under the supervision of Dr. Christine Kraus on the SNO+ experiment. During my term I spent a great deal of time engineering mechanical systems for the project. Specifically I cleverly fixed the slippage issues plaguing the umbilical retrieval mechanisms (URM's), a device responsible for lowering radioactive sources into the multi-million dollar detector. In addition to redesigning high-traction pulleys, I also designed and fabricated a chain-drive system which successfully met budget, space and radioactivity requirements. My time with SNOLAB has given me a great mechanical intuitiveness that I believe combined with my research experience and knowledge of physics will give me an advantage in engineering science.

I have completed a number of projects involving delicate, technical design and precise assembly. For a directed studies project I designed and professionally constructed operational sensors for a high altitude balloon using a LPKF circuit milling machine and Eagle CAD drafting software. The circuits included a 900MHz 1W transmitter/reciever, GPS telemetry APRS transmitter, a cosmic radiation detector, accelerometer and atmospheric sensors all designed as Arduino shields. The Project required a great deal of electronic design and debugging. This experience combined with my education and hardworking nature will be valued in engineering science.

I hope to hear from you soon to discuss how I can become a valued member of your institution. I look forward to starting my engineering career, as well as pursuing opportunities for professional development and advancement within Simon Fraser University. Thank you for your kind consideration of my qualifications.

\makeletterclosing % Print letter signature

%----------------------------------------------------------------------------------------

\end{document}